%%%%%%%%%%%%%%%%%%%%%%%%%%%%%%%%%%%%%%%%%
%  Iscp Handbook EE department 
% The original template (the Legrand Orange Book Template) can be found here --> http://www.latextemplates.com/template/the-legrand-orange-book
%
% Original author of the Legrand Orange Book Template:
% Mathias Legrand (legrand.mathias@gmail.com) with modifications by:
% Vel (vel@latextemplates.com)
%
% Original License:
% CC BY-NC-SA 3.0 (http://creativecommons.org/licenses/by-nc-sa/3.0/)
% just a mere adaptation of https://www.overleaf.com/articles/clustering-the-interstellar-medium/mtthgyyfrdkn
%%%%%%%%%%%%%%%%%%%%%%%%%%%%%%%%%%%%%%%%%
 
%----------------------------------------------------------------------------------------
%	PACKAGES AND OTHER DOCUMENT CONFIGURATIONS
%----------------------------------------------------------------------------------------

\documentclass[openany]{book} % Default font size and left-justified equations

\usepackage[top=3cm,bottom=3cm,left=3.2cm,right=3.2cm,headsep=10pt,letterpaper]{geometry} % Page margins

\usepackage{xcolor} % Required for specifying colors by name
\definecolor{ocre}{RGB}{52,177,201} % Define the orange color used for highlighting throughout the book

% Font Settings
\usepackage{avant} % Use the Avantgarde font for headings
%\usepackage{times} % Use the Times font for headings
\usepackage{mathptmx} % Use the Adobe Times Roman as the default text font together with math symbols from the Sym­bol, Chancery and Com­puter Modern fonts

\usepackage{microtype} % Slightly tweak font spacing for aesthetics
\usepackage[utf8]{inputenc} % Required for including letters with accents
\usepackage[T1]{fontenc} % Use 8-bit encoding that has 256 glyphs

\usepackage[object=vectorian]{pgfornament} %for ornamental lines 
\newcommand{\sectionlinetwo}[2]{%
  \nointerlineskip \vspace{.5\baselineskip}\hspace{\fill}
  {\resizebox{0.5\linewidth}{1.2ex}
    {\pgfornament[color = #1]{#2}
    }}%
    \hspace{\fill}
    \par\nointerlineskip \vspace{.5\baselineskip}
  }
\usepackage{tikz}
\newcommand{\photo}[3]{%
	\tikz\node[circle,draw,inner sep=#1,text=white,path picture={\node at (path picture bounding box.center){\includegraphics[width=#2]{#3}};}]{};
}%

% Bibliography
\usepackage[style=alphabetic,sorting=nyt,sortcites=true,autopunct=true,babel=hyphen,hyperref=true,abbreviate=false,backref=true,backend=biber]{biblatex}


\usepackage[colorlinks = true,
	    linkcolor = blue,
	    urlcolor  = blue,
	    citecolor = blue,
	    anchorcolor = blue]{hyperref}
\usepackage{miama}
\usepackage[T1]{fontenc}

\usepackage{csvsimple} % for reading data directly from csv
\begin{filecontents*}{./iscp/test.csv}
imgA,nameA,emailA,imgB,nameB,emailB,imgB,nameC,emailC,imgD,nameD,emailD
\end{filecontents*}

\addbibresource{bibliography.bib} % BibTeX bibliography file
\defbibheading{bibempty}{}

\input{structure} % Insert the commands.tex file which contains the majority of the structure behind the template

\begin{document}
\title{EE Handbook ISCP}

%----------------------------------------------------------------------------------------
%	TITLE PAGE
%----------------------------------------------------------------------------------------

\begingroup
\thispagestyle{empty}
\AddToShipoutPicture*{\put(0,0){\includegraphics[scale=0.42]{./pictures/iitbombayTwire.jpg}}} % Image background
\centering
\vspace*{5cm}
\par\normalfont\fontsize{35}{35}\sffamily\selectfont
\textbf{Department Handbook}\\
{\LARGE Electrical Engineering}\par % Book title
{\Huge 2019-2020 }\par % Author name
\endgroup

%----------------------------------------------------------------------------------------
%	DISCLAIMER PAGE
%----------------------------------------------------------------------------------------

\newpage
~\vfill
\thispagestyle{empty}

%\noindent Copyright \copyright\ 2014 Andrea Hidalgo\\ % Copyright notice

\noindent \textsc{DISCLAIMER}\\

%\noindent \textsc{github.com/LaurethTeX/Clustering}\\ % URL

\noindent The Institute Student Companion Program (ISCP) has acquired and presented the data in this
handbook on a best effort basis. However the correctness of the information is not guaranteed.
ISCP will not be held responsible for any inaccuracies in the document.\\

\noindent \textit{First release, May 2018} % Printing/edition date

%----------------------------------------------------------------------------------------
%	TABLE OF CONTENTS
%----------------------------------------------------------------------------------------

\chapterimage{./pictures/tableofcontents.jpg} % Table of contents heading image

\pagestyle{empty} % No headers

\tableofcontents % Print the table of contents itself

%\cleardoublepage % Forces the first chapter to start on an odd page so it's on the right

\pagestyle{fancy} % Print headers again

%----------------------------------------------------------------------------------------
%	CHAPTER 1
%----------------------------------------------------------------------------------------

\chapterimage{./pictures/convo.jpg} % Chapter heading image

\chapter{About The Department}


\section*{}
The Department of \href{https://www.ee.iitb.ac.in/web}{\color{blue}{Electrical Engineering (EE)}} is one of the largest departments of IIT Bombay since its inception in 1958. The department has different academic programs with about 570 undergraduate and 730 postgraduate students. The department is equipped with the state of the art experimental and computational facilities for undertaking R $\&$ D and consultancy activities in various fields.\\
\\
The EE department has a vibrant postgraduate program with strong focus on research and development. The number of postgraduate students in EE is more than that of the undergraduate students and the department attaches a lot of importance to its Masters' students as they constitute the backbone of research and development.
\\
\\The department offers M.Tech in six research areas / specializations:
\begin{itemize}
    \item Communications Engineering (EE1)
    \item Control and Computing (EE2)
    \item Power Electronics and Power Systems (EE3)
    \item Electronic Systems (EE5)
    \item Integrated Circuit and Systems (EE6)
    \item Solid State Devices (EE7)
\end{itemize}
\medskip
Faculty members of the department are recipients of many distinguished awards like Shanti Swarup Bhatnagar Prize, Prof. K. Sreenivasan Memorial Award, Prof. SVC Aiya Memorial Award, Dr. Vikram Sarabhai Research Award, Ram Lal Wadhwa Award, INAE Young Engineer Award, Alexander von Humboldt Fellowship and many others.\\
\\
Many faculty members are Editors of IEEE and other national and international journals. They are also Fellows of organizations like IEEE, IETE, INAE, IASc, NASI and INSA.
%This statement requires citation \cite{book_key}; this one is more specific \cite[122]{article_key}.
%----------------------------------------------------------------------------------------
%	CHAPTER 2
%----------------------------------------------------------------------------------------
\chapterimage{./pictures/bgf4.png}

\chapter{Message from H.O.D}

\section*{}
Congratulations on your selection for the M.Tech/Ph.D. program in EE at IIT Bombay. As you know, the competition was very stiff and you are among the very few students who made it. We, the faculty members, staff and students extend a warm welcome to you.
%\begin{figure}[h]
    %\centering
    %\includegraphics[width=0.77\textwidth]{ha-gray-conv-crp.jpg}
    %\caption{Picture of the M83 galaxy, image taken from the WFC3 ERS M83 Data Products, http://archive.stsci.edu/prepds/wfc3ers/m83datalist.html}
    %\label{fig:awesome_image}
%\end{figure}
\medskip
\newline As you are aware, ours is among the largest Electrical Engineering departments in the country with 64 faculty members and 1360 students, of which more than 55\% are graduate student. We have a strong academic and research culture. We have state-of-the-art research laboratories in almost all areas of electrical engineering and a few centres of excellence. I am sure you will find this place academically rewarding.
\medskip
\newline Your department has a lot to give –just how much you take depends on one person-and that is you. You will face diverse temptations but you need to stay focused to achieve your goals. Do good work-you grow, and the department does too. It is a win-win situation. All in all, I assure you the time spent here will be the best years of your life!
\begin{center}
    \textbf{Feel free to contact me if you need any help!}\\

\begin{flushright}
	{\LARGE\fmmfamily
	Baylon G Fernandes\\}
	Head Of Department\\
	bgf@ee.iitb.ac.in\\
\end{flushright}
\end{center}
%----------------------------------------------------------------------------------------
%	CHAPTER 3
%----------------------------------------------------------------------------------------


\chapterimage{./pictures/iscp_bg.png}
\chapter{Message from ISCP}

\section*{}

Dear New Entrants,\\
We take this opportunity to welcome you to one of the most prestigious institutes of the country. We
congratulate you on having achieved this feat. With our personal experiences we can vouch that the your stay here at the campus would be exciting. From potential leopard sightings to potential bumping into movie stars all awaits you. Wonderful all night banters, amazing wing cultures and mad birthday celebrations are a few things that you will carry from here when you leave, obviously along with the degree. You will also become a part of a culture where people want to perfect their craft and thus work day in and day out at it. Hence there will be great opportunities to learn both inside as well as outside the classrooms. Thus it is a whole new cosmos to enter and with excitements it might have a few challenges too. We at \href{https://gymkhana.iitb.ac.in/~scp/scp/index.html}{\color{blue}{Institute Students Companion Program (ISCP)}}  work towards providing you with the hacks to take care of these challenges and have a happy stay with the IIT Bombay family.\\
\\
The primary objective of the Companion Programme under which the ISCP team works is to build a
relationship of trust and comfort between the final year students and the incoming students of the PG
programmes. Once this is established life at campus becomes so much easier than what it would have been
without it. The knowledge and the experience that the senior batch has gained with their stay at the campus makes the surroundings so much you that the transition becomes smooth. From the lingo on campus to the terminology` in the classroom, from the grading to the syllabus, from the profile to placements, from tagda franky to bhindi rice all becomes ingrained so much as if it were you always.\\ 
\\
On campus you might always be short on time as there is so much to do and when there is so much to do time
flies at sonic speeds. Managing the academics along with extra curricular activities and your social life may seem a daunting task at times. The ISCP programme thus provides you a Student Companion with whom you
can share your academic and non- academic problems. These are self-motivated volunteers who want to
genuinely help you in tough situation as a giving back act of what they received from the programme.\\  
\\ You can look up to the team for any initial information in things that you are venturing out at be it academics or extra-curriculars, any academic or non- academic issues that you are facing, any sort of support, any requirement that you wanna raise up as a part of the student community and last but certainly not the least just for normal interaction because that is all the programme holds at its core.\\
\\
Come be a part of this immense pool of wisdom and make it more happening and diverse. \\

 What to expect from a Student Companion:
	\begin{itemize}
	    \item Initial information about the campus, courses, academics and extracurricular activities.
	    \item Support in case of any problem or difficulty.
	    \item Organization of various academic and non-academic activities for student’s development.
	    \item Continuous interaction and feedback from students on their needs and requirements.
	\end{itemize}
 In short, this is a program by the PG students of IIT Bombay for the new entrants to ensure their easy adaptation to IIT Bombay culture and assist them in an overall development through utilization of all the available resources at IIT Bombay. Let the learning begin. Feel free to contact us anytime!\\
\begin{center}
    Mail to: iscp@iitb.ac.in\\
Overall Coordinators\\
Institute Student Companion Programme (2019-20)\\
\begin{center}
	\begin{tabular}{ccc}
		\photo{1cm}{35mm}{./pictures/uroosa.jpeg}
		& \photo{1cm}{35mm}{./pictures/tumul.jpeg} 
		& \photo{1cm}{35mm}{./pictures/avinash.jpeg} \\
		 \textbf{Uroosa Warsi}
		&\textbf{Tumul Rai}
		&\textbf{ Avinash Singh}\\
		\textit{uroosawarsi134@gmail.com}
		&\textit{tumulrai91@gmail.com}
		& \textit{Avinashindolia007@gmail.com}\\
		\textbf{7835877634}
		&\textbf{7275362979}
		&\textbf{+919058076777}\\
		\end{tabular}
%Uroosa Warsi | Tumul Rai | Avinash Singh\\
%+91-9007766390 | +91-9432152174 | +91-7008955255
%\vfill
	\\
	\pgfornament[color=black]{89}
	 % \foreach \i in {1,...,99} {\expandafter\pgfornament\expandafter{\i}\ \i \\ }
 \end{center}
 \end{center}
~\vfill
%----------------------------------------------------------------------------------------
%	CHAPTER 4
%----------------------------------------------------------------------------------------

\chapterimage{./pictures/iscp_dep_bg.png} % Chapter heading image

\chapter{Message from Department Coordinators}
\begin{flushleft}

Hello Junta!!\\
\end{flushleft}
Congratulations on becoming a part of the IIT Bombay family. We are delighted to welcome you to one of the best departments of the premier institutes in the world. This institute will provide you with an engaging, vibrant, and inclusive community of learners. From your first day to the last, you will be amazed by the array of extraordinary faculties, outstanding facilities and helpful staff in our department which together provides an excellent biosphere for learning.\\
However, the institute’s greatest and most enduring strength is the balance between the quality of education and the extra-curricular activities. It is assured that you will get a lot of opportunities on the academic as well as extra-curricular front to acquire new skills, nurture and enhance your dormant talents and build competence to take on the world.\\
Once again, a warm welcome from all the seniors. Have a wonderful stay at IIT Bombay!!!\\
\begin{center}
	\begin{tabular}{cc}
		\photo{1cm}{35mm}{./pictures/sunny.jpeg}
		& \photo{1cm}{35mm}{./iscp/ram.jpg} \\
		 \textbf{Sunny Mehta}
		&\textbf{Ramkrishna}\\
		\textit{sunnymehta78669@gmail.com}
		&\textit{ramporicha@ee.iitb.ac.in}\\
		+919677165155 & +919571648532

	\end{tabular}
\end{center}
\sectionlinetwo{black}{88}

\chapterimage{./pictures/imr_bg.png} % Chapter heading image

\chapter{Message from IMR}

\section*{}
Hello, Friends!\\
	Congratulations on being selected to be a part of IIT Bombay and a cordial welcome to this new world.\\
You must have realized by now that you are undergoing a phase of substantive transformation, and this might be daunting for some of you. The academic curriculum of this institution might seem different and perhaps new in comparison to what you were exposed to during your undergraduate education. To appease all the apprehensions that you have, the Post Graduate Academic Council (PGAC) along with the team of (ISCP) will try to address all the queries that you will be having during your entire Master’s Programme. \\
IIT Bombay is known to offer the students a very dynamic environment and a reasonable amount of freedom so that the individual can pursue their heart’s desire. Be it academics, sports, cultural, or any other activity, you will be finding myriad opportunities to build up your personality and add value to your life. I am hoping that you will be able to explore the unending map of possibilities, push your boundaries, break all the walls and bring out the best version of yourself by the time you finish your Degree Programme.\\
I wish you good luck and hope to see you around. We are looking forward to interacting with you. \\
\\
\begin{center}
	\begin{tabular}{c}
		\photo{1cm}{35mm}{./pictures/issa.jpeg} \\
		\textbf{Himanshu Bishwash}\\
		\textbf{imr@iitb.ac.in}\\
		\textit{8698965842/9867110738}\\
		\end{tabular}
\end{center}

%----------------------------------------------------------------------------------------
%	CHAPTER 5
%----------------------------------------------------------------------------------------


\chapterimage{./pictures/iscp_bg.png}
\chapter{LAB Facilities}

\chapterimage{./pictures/iscp_bg.png}
\begin{enumerate}
    \item \href{https://www.ee.iitb.ac.in/~wel_iitb/}{\color{blue}  \textbf{Wadhwani Electronics Laboratory (WEL)}}\\
    (\textit{3rd Floor ,  Department of Electrical Engineering})\\
    Professor In-charge - Siddharth Tallur\\
Relevant Specializations - for all specializations in EE\\
\\
The WEL houses all the major electronics hardware activities of the Electrical Department at IIT Bombay. Around 12 major laboratory course are conducted every year in WEL. Around 650 UG students and 250 PG and DD students enroll for lab courses annually. In addition to courses, WEL is also home to all major electronics projects done by students at IIT Bombay. Various Technical events are housed here, including electronics workshops,competitions, teachers' training etc.\\

\item \href{http://www.ee.iitb.ac.in/~spilab} {\color{blue}  \textbf{Signal Processing and Instrumentation Lab} }\\
    (\textit{1st Floor ,  Department of Electrical Engineering})\\
    Professor In-charge - P.C Pandey\\
Relevant Specializations - EE1 , EE5\\
\\
This lab focuses on research in the areas of speech signal processing, bio-medical signal processing & instrumentation, electronic instrumentation and embedded system design. The research problems being tackled by students in the lab include 'Enhancement of electroaryngeal speech using spectral subtraction' , 'Multi-band frequency compression for hearing impaired' , 'Speech enhancement by modification of stop consonant landmarks', 'Diagnostic information from impedance cardiograph' etc.\\


\item \href{https://www.ee.iitb.ac.in/web/research/labs/isl} {\color{blue} \textbf{Integrated Systems Laboratory}}\\
    (\textit{1st Floor, Electrical Annex Building, Opposite to EE main building})\\
    Professor In-charge - Prof. Jayanta Mukherjee, Prof. Maryam Shojaei Baghini\\
Relevant Specializations - EE1 , EE5 , EE6 , EE7\\
\\
Integrated Systems Laboratory at IIT Bombay is a simulations and testing laboratory located in the first floor of the EE Annex Building. Embedded system solutions are developed here. Primarily design and test of passive and active RF and circuits is done. EM design software like CST Microwave Studio, HFSS and Key sight ADS are installed in lab. Several Core i7 PC's with augmented RAM's to enable high end computing are also present. GPU based high end workstations for enabling fast EM solvers for design at millimetre wave frequencies are available in lab. This lab work closely with the VLSI lab for RFIC simulations.\\


\item \href{https://www.ee.iitb.ac.in/web/research/labs/emsys} {\color{blue} \textbf{Embedded System lab}}\\
    (\textit{5th floor, Girish Gaitonde (GG) building})\\
    Professor In-charge - Prof. Maryam Shojaei Baghini, Prof. Madhav P. Desai\\
Relevant Specializations - EE5 , EE6 , EE7\\
\\
The prime focus areas of lab are system design, prototyping and evaluation starting from sensor/transducer interfacing to full system development and network of sensor nodes, signal processing system design, analysis and implementation for various sensing-based applications, AI for ASIC design, sensor nodes and networks, deep learning for signal and data Analysis, hardware-accelerated simulation. Research work is also done in agricultural, bio-medical domains.\\

\item \href{http://www.ee.iitb.ac.in/~stallur/index.php/} {\color{blue} \textbf{Applied Integrated Micro Systems (AIMS) Laboratory}}\\
    (\textit{1st Floor, Electrical Annex Building, Opposite to EE main building})\\
    Professor In-charge - Siddharth Tallur\\
Relevant Specializations - EE5 , EE6 , EE7\\
\\
AIMS lab works on  innovative instrumentation for impactful measurements. The research areas include sensor systems, hybrid integrated microsystems,  studying their underlying physics  to leverage such platforms for high resolution sensing applications.\\

\item \href{http://www.ee.iitb.ac.in/~kasturis/index.php} {\color{blue} \textbf{Photonics and Quantum Enabled Sensing Technology (P-Quest) Laboratory}}\\
    (\textit{2nd Floor, Department of Electrical Engineering})\\
    Professor In-charge - Kasturi Saha\\
Relevant Specializations - EE1 , EE7\\
\\
P-Quest lab works on exploring precision metrology and sensing using novel interdisciplinary research in fields like nano-photonics, classical and quantum information processing and life sciences, to develop practical quantum devices via design and experimentation, thus connecting quantum theory to engineering applications.\\

\item \href{https://www.ee.iitb.ac.in/vlsi/} {\color{blue} \textbf{VLSI Design Lab}}\\
    (\textit{5th Floor, Girish Gaitonde (GG) Building})\\
Relevant Specializations - EE5 , EE6\\
\\
The VLSI Design Lab hosts all major VLSI CAD Vendor’s tool and their licenses .Few major tools in frequent use are by Synopsys, Cadence, Mentor, Agilent, Magma, Xilinx etc. The main research focus is in the area of analog and digital design. In addition to courses, this lab also hosts accounts for different courses which require hands on experience of tools. For different projects and their tape-out, there is availability of a high performance computational server to speed up the simulations.\\

\item \href{https://www.ee.iitb.ac.in/~spann/} {\color{blue} \textbf{Signal Processing and Artificial Neural Networks (SPANN)}}\\
    (\textit{3rd Floor, Department of Electrical Engineering})\\
    Professor In-charge - S.N. Merchant\\
Relevant Specializations - EE1\\
\\
The major areas of research which are pursued in SPANN Lab include Wireless Communications, Sensor Networks, Image Processing and Signal Processing.\\
\\
\item \href{https://www.ee.iitb.ac.in/~infonet/} {\color{blue} \textbf{Information Networks Laboratory}}\\
    (\textit{2nd Floor, Department of Electrical Engineering})\\
    Professor In-charge - Prasanna Chaporkar and Abhay Karandikar\\
Relevant Specializations - EE1\\
\\
Group members of the lab are pursuing research in the field of 4G and 5G cellular technologies, with an emphasis on inter-working with non-3GPP Wireless Local Area Networks (WLANs). Other important focus areas of the lab include-mechanisms for spectrum sharing and multi-cast for 4G networks, integrated access and backhaul systems, multi-connectivity in 5G networks and Software Defined Networking (SDN) for cellular networks and WLAN. The TTSL - IITB Center of Excellence in Telecommunication (TICET) facility is also a part of this lab.\\

\item \href{http://www.ee.iitb.ac.in/~tidsplab/index.php} {\color{blue} \textbf{Texas Instruments Digital Signal Processing Lab (TIDSP)}}\\
    (\textit{3rd Floor, Department of Electrical Engineering})\\
    Professor In-charge - V.M. Gadre\\
Relevant Specializations - EE1 , EE5\\
\\
TIDSP laboratory was set up in the EE Department to support DSP hands-on projects at the undergraduate and the postgraduate levels. DSP specific hardware and software support is provided by Texas Instruments (TI) itself.\\

\item \href{https://www.ee.iitb.ac.in/~foclab/} {\color{blue} \textbf{Fiber-Optics Communication Lab}}\\
    (\textit{2nd Floor, Department of Electrical Engineering})\\
    Professor In-charge - Kumar Appaiah , Joseph John\\
Relevant Specializations - EE1 , EE5\\
\\
This lab is dedicated to pursue research mainly in the area of optical fiber communication ( SM, MM, FM), plastic optical fiber and fiber sensing.\\

\item \href{https://www.ee.iitb.ac.in/bharticentre/index.html} {\color{blue} \textbf{Bharti Centre for Communication}}\\
    (\textit{2nd Floor, Department of Electrical Engineering})\\
    Professor In-charge - D. Manjunath , Bikash Kumar Dey\\
Relevant Specializations - EE1\\
\\
The Bharati Centre for Communication is a centre to generate fundamental knowledge in telecommunication and allied systems. The Vision of the centre is to  be an internationally recognised contributor in moving the frontiers of knowledge through research and education, to keep technology practise in focus and to educate for innovation and leadership.\\

\item \textbf{Vision and Image Processing}\\
    (\textit{ , Department of Electrical Engineering})\\
    Professor In-charge - \\
Relevant Specializations - \\
\\
This lab is dedicated to Deep Learning, Computer vision techniques. The major projects currently undertaken are related to Haptics, Biometrics, Image segmentation, super-resolution, Anomaly detection and surveilling related problems. This lab consists of more than 23 GPU, and high-performance computer to work on the mentioned techniques. Research is currently heading in the direction of surveillance such as aerial and single camera view. The task involves are human pose estimation, scene understanding, object detection, etc.\\

\item  \textbf{PC Lab}\\
    (\textit{1st Floor, Department of Electrical Engineering})\\
    Professor In-charge - \\
Relevant Specializations - For all specializations in EE\\
\\


\item  \textbf{Digital Audio Processing Lab}\\
    (\textit{,, Department of Electrical Engineering})\\
    Professor In-charge - \\
Relevant Specializations -5\\
\\
This lab is based on the application of signal processing in the analysis of speech and audio. Research activities are related to spoken language assessment, music content analysis, measuring the goodness of instruments like Tabla, segmentation of instruments in the music concert and other application of speech and audio processing.\\




\item \textbf{Communication Lab}\\
    (\textit{1st Floor, Department of Electrical Engineering})\\
    Professor In-charge - Shalabh Gupta\\
Relevant Specializations - EE1 , EE5 , EE6\\
\\
Communication Lab primarily focuses upon cutting 
edge research in the area of High speed Communication 
Links. It can further be divided into different domains like  High speed Links using Optical Communication, 
Silicon Photonics, SerDes (Serialiser and Deserialiser) Links, RF Circuits and Millimetre-wave circuits and Systems.  Besides it is also working in the domain of RF Electronics,  Embedded Systems and Audio Video Communication for some of its projects.\\

\item  \textbf{Networking Lab}\
    (\textit{2nd Floor, Department of Electrical Engineering})\\
    Professor In-charge - D. Majunath\\
Relevant Specializations - EE1\\
\\
The work in the Networking lab deals with the theoretical aspects of queuing theory, sensor networks, applications of stochastic approximation, software routing etc. \\

\item \href{https://www.ee.iitb.ac.in/~mwave/} {\color{blue} \textbf{Microwave and Antenna Lab}}\\
    (\textit{3rd Floor, Department of Electrical Engineering})\\
    Professor In-charge - Girish Kumar\\
Relevant Specializations - EE1\\
\\
Microwave Lab is involved in research work in the area of RF Systems, Electromagnetic Waves and Antenna Design. Primary research work is being done in different fields like Micro-strip Antenna, Microwave Integrated Circuits and Broadband Antennas.\\

\item \href{https://www.ee.iitb.ac.in/web/research/labs/ticet} {\color{blue} \textbf{TTSL-IITB Centre of Excellence in Telecommunications (TICET)}}\\
    (\textit{2nd Floor, Department of Electrical Engineering})\\
    Professor In-charge - \\
Relevant Specializations - EE1\\
\\
TICET focuses on state of art research in telecom relevant to Indian Service Providers in general and Tata Teleservices Limited (TTSL) in particular with special emphasis on rural wireless applications and connectivity. The research activities in this lab are related to Quality of Service and resource allocation in wired/wireless networks,TV White Space and its potential for affordable broadband access in India, Frugal 5G and rural broadband research and standardization.\\

\item \href{https://www.ee.iitb.ac.in/~asethi/research.html} {\color{blue} \textbf{Medical Deep learning and AI Lab (MeDAL)}}\\
    (\textit{1st Floor, Department of Electrical Engineering})\\
    Professor In-charge - Amit Sethi , Manoj Gopalkrishnan\\
Relevant Specializations - EE1\\
\\
This lab is dedicated towards solving real world problems in the areas of medical imaging, radiology and pathology using deep learning architechtures. This lab houses high end computing facilities to work with large scale data (Giga pixel images) to solve various computer vision problems. Research group have colloboration with various hospitals and universities. Some ongoing engagements are with TATA memorial hospital, University of Illinois and King's college London.\\

\item \textbf{Information Systems and Radios (ISR) Lab}\\
    (\textit{2nd Floor, Department of Electrical Engineering})\\
    Professor In-charge - Sibi Raj Pillai\\
Relevant Specializations - EE1\\
\\
\\
\end{enumerate}
   

\chapter{Faculty Members}
\section{EE1: Communication and Signal Processing}
\begin{tabular}{p{3.5cm} p{3.5cm}p{9cm}}
\hline 
\hline 
Faculty  & Office  & Research Interests \\ 
\hline

\href{https://www.ee.iitb.ac.in/wiki/faculty/chaporkar}{\color{blue}{Prof. Prasanna Chaporkar }} & • & Resource Allocation and scheduling in wired/wireless networks, Optimization and control of stochastic systems, Distributed systems and algorithms \\ 
\hline 
\href{https://www.ee.iitb.ac.in/~sc/}{\color{blue}{Prof. Subhasis Chaudhuri}} & • & Multimedia, Computer Vision, Image Processing, Pattern Recognition,Biomedical Signal Processing, Computational Haptics \\ 
\hline 
\href{https://www.ee.iitb.ac.in/wiki/faculty/vmgadre}{\color{blue}{Prof. Vikram M. Gadre }}& • & Communication and signal processing, with emphasis on multiresolution
and multi-rate signal processing, especially wavelets and filter banks:
theory and applications \\ 
\hline 
\href{https://www.ee.iitb.ac.in/wiki/faculty/shalabh}{\color{blue}{Prof. Shalabh Gupta}} & • & High-speed CMOS analog/RF/mm-wave integrated circuits and systems,Optical fiber communication systems, Microwave photonics / ultrafast data  conversion using photonics , Beam forming antenna systems, Signal processing for these systems \\ 
\hline 
\href{https://www.ee.iitb.ac.in/wiki/faculty/jjohn}{\color{blue}{Prof. Joseph John}} & • & Analog and Digital Circuits, Optical Fiber Communications, Indoor Optical Wireless Systems, Modern Electronic Systems and Instrumentation \\ 
\hline 
\href{http://www.iitb.ac.in/en/employee/prof-abhay-karandikar}{\color{blue}{Prof. Abhay Karandikar }}& • & Control and Performance Modeling of Wireless Networks, Quality of Service and Resource Allocation in Wired/Wireless Networks, Next Generation Wireless Network Protocols (related to 802.16m, LTE-Advanced and 4G Standards), Co-operative Relay and Self Organizing Network,Carrier Ethernet and Mobile Backhaul, Rural Wireless Network \\ 
\hline 
\href{https://www.ee.iitb.ac.in/~gskasbekar/}{\color{blue}{Prof. Gaurav S. Kasbekar}}& • & Modeling, design and analysis of wireless networks, Game theoretic and economic aspects of spectrum allocation, Cognitive radio networks,
Lifetime and coverage problems in wireless sensor networks \\ 
\hline 
\href{https://www.ee.iitb.ac.in/~animesh/}{\color{blue}{Prof. Animesh Kumar }} & • & Signal processing, communication systems, applied statistics, SRAM reliability models \\ 
\hline 
\end{tabular} 
\begin{tabular}{p{3.5cm} p{3.5cm}p{9cm}}
\hline 
\hline 
\href{https://www.ee.iitb.ac.in/wiki/faculty/gkumar}{\color{blue}{Prof. Girish Kumar }} & • & Microstrip antennas and arrays, Broadband antennas, Microwave integrated circuits, EMI/ EMC, RF communication circuits \\ 
\hline 
\href{https://www.ee.iitb.ac.in/~dmanju/}{\color{blue}{Prof. D. Manjunath }}& • & Computer and Communication Network Protocols, Systems and Algorithms Performance Modeling, Queueing Theory and Simulation, Stochastic
Systems \\ 
\hline 
\href{https://www.ee.iitb.ac.in/wiki/faculty/merchant}{\color{blue}{Prof. Shabbir Merchant }}& • & Signal Processing, Adaptive Signal Processing \\ 
\hline 
\href{https://www.ee.iitb.ac.in/~pcpandey/}{\color{blue}{Prof. Prem C. Pandey }}& • & Speech and Signal Processing, Biomedical Signal Processing and Instrumentation,Electronic Instrumentation, Embedded Electronic System Design \\ 
\hline 
\href{https://www.ee.iitb.ac.in/wiki/faculty/bsraj}{\color{blue}{Prof. Sibi Raj B Pillai  }}& • & Fundamental Limits of Communication Systems, Information Theory and its applications, Compressed Sensing, Stochastic Modeling, Resource Allocation Problems, Interference Channels, Relaying and Broadcasting \\ 
\hline 
\href{https://www.ee.iitb.ac.in/wiki/faculty/bikash}{\color{blue}{Prof. Bikash Kumar Dey }}& • & Information Theory, Coding Theory, Wireless communication. \\ 
\hline 
\href{https://www.ee.iitb.ac.in/wiki/faculty/prao}{\color{blue}{Prof. Preeti Rao }}& • & Speech and Audio Signal Processing, Music Information Retrieval \\ \hline 
\href{https://www.ee.iitb.ac.in/web/faculty/homepage/rajbabu}{\color{blue}{Prof. Rajbabu Velmurugan }}& • & Statistical and digital signal processing, Signal processing system design, Particle filter applications, Target tracking systems \\ 
\hline 
\href{https://www.ee.iitb.ac.in/~sarva/}{\color{blue}{Prof. Saravanan Vijayakumaran}} & • & Signal Processing for Communications, Parallel Simulation Algorithms \\ 
\hline 
\href{https://sites.google.com/site/nikhilkaram/}{\color{blue}{Prof. Nikhil Karamchandani }}& • & Information Theory, Networks, Communications,Distributed Computation, Cyber-Physical Systems. \\ 
\hline 
\href{https://www.ee.iitb.ac.in/~akumar/}{\color{blue}{Prof. Kumar Appaiah}}& • & Signal processing for communication, fibre optics, wireless communication. \\ 
\hline 
\href{https://www.ee.iitb.ac.in/~jayakrishnan.nair/}{\color{blue}{Prof. Jaykrishnan U. Nair }}& • & Queueing theory, Communication networks, Heavy tails. \\ 
\hline 
\href{https://www.ee.iitb.ac.in/~manojg/}{\color{blue}{Prof. Manoj Gopalkrishnan}} & • & Algorithms in nature,Information processing in networks,Reaction networks,Neural networks,Evolution,Game theory,Deep learning,Information geometry,Thermodynamics of information,Quantum Information\\ 
\hline 
\href{https://www.ee.iitb.ac.in/~asethi/}{\color{blue}{Prof. Amit Sethi }}& • & Computational pathology,Medical image analysis,Deep learning,Machine learning,Computer vision,Image processing,Signal processing \\ 
\hline 
\href{https://sites.google.com/site/sharayumoharir/}{\color{blue}{Prof. Sharayu Moharir }}& • & Modeling and the design of scalable resource allocation algorithms for large networks, including content delivery networks, communication
networks and crowd-sourcing. \\ 
\hline 
\end{tabular} 
\section{EE2: Control and Computing}
\begin{tabular}{p{3.5cm} p{3.5cm}p{9cm}}
\hline 
\hline  
Faculty  & Office  & Research Interests \\ 
\hline 
\href{https://www.ee.iitb.ac.in/wiki/faculty/belur}{\color{blue}{Prof. Madhu N. Belur}} & Room 237A, Second floor, main EE building, Department of Electrical Engineering & Control theory, dissipative systems, graph theoretic methods, decentralized control, behavioral theory control, Fault diagnosis \\
\hline 
\href{https://www.ee.iitb.ac.in/web/faculty/homepage/borkar}{\color{blue}{Prof. Vivek Shripad Borkar }}& EA202, EE Annexe, Department of Electrical Engineering & Stochastic Control, Learning Control Theory, Random Processes \\ 
\hline 
\href{https://www.ee.iitb.ac.in/wiki/faculty/dc}{\color{blue}{Prof. Debraj Chakraborty }}& New Office Complex, 1st floor EE (next to PC Lab), Department of Electrical Engineering & Optimal Control, Linear Systems, Optimization, Differential Games,Game Theory \\ 
\hline 
\href{https://www.ee.iitb.ac.in/wiki/faculty/chaporkar}{\color{blue}{Prof. Prasanna Chaporkar }}& Room 231B,Second floor,main EE building,Department of Electrical Engineering & Resource Allocation and scheduling in wired/wireless networks, Optimization and control of stochastic systems, Distributed systems and algorithms \\ 
\hline 
\href{https://www.ee.iitb.ac.in/~sc/}{\color{blue}{Prof. Subhasis Chaudhuri }}& Room 231A,Second floor,main EE building,Department of Electrical Engineering & Multimedia, Computer Vision, Image Processing, Pattern Recognition,Biomedical Signal Processing, Computational Haptics \\ 
\hline 
\href{https://www.ee.iitb.ac.in/wiki/faculty/svk}{\color{blue}{Prof. Shrikrishna V. Kulkarni}} & 211-C, Department of Electrical Engineering & Transformers: Design, Analysis and Diagnostics, Electromagnetic and Coupled Field Computations, Power Engineering: Distributed Generation, High Voltage Engineering: Insulation Design/Diagnostics \\ 
\hline 
\href{https://www.ee.iitb.ac.in/~debasattam/}{\color{blue}{Prof. Debasattam Pal}} & Room No 231-D 2nd Floor EE Main Building & Distributed parameter systems, algebraic analysis, optimal control. \\ 
\hline 
\href{https://www.ee.iitb.ac.in/wiki/faculty/vrs}{\color{blue}{Prof. Virendra R. Sule }} & New Office Complex, 1st floor EE (next to PC Lab), Department of Electrical Engineering & Cryptology: Block and stream ciphers, efficient arithmetic for public key cryptography, algebraic cryptanalysis, Dynamical systems and feedback control theory \\ 
\hline
\href{https://scholar.google.co.in/citations?user=9lWahYMAAAAJ&hl=en}{\color{blue}{Prof. Dwaipayan Mukherjee }}& Room no EE 214D, 2nd floor EE  Main Building & Multi-agent Systems,Consensus,Formation Control,Control Theory and Robust Control \\ 
\hline 

\href{https://www.ee.iitb.ac.in/wiki/faculty/hp}{\color{blue}{Prof. Harish K. Pillai }} & 231-D,EE Main Building 231-D,EE Main Building & Control theory, Systems theory, Multidimensional systems, Numerical
and computational methods, Coding theory, Optimization techniques, Electromagnetics \\ 
\hline
\end{tabular}
\vfill
\section{EE3: Power Electronics and Power Systems}
~\vfill
~\vfill
\begin{tabular}{p{3.5cm} p{3.5cm}p{9cm}}
\hline 
\hline 
Faculty  & Office  & Research Interests \\ 
\hline 
\href{https://www.ee.iitb.ac.in/~agarwal/}{\color{blue}{Prof. Vivek Agarwal }}& • & Power conversion: New converter topologies, High frequency link
power conversion, ZCS-ZVS configurations, Switched Capacitor DCDC converters, Power quality issues: Power factor correction techniques, Static VAR compensation, Active filters \\ 
\hline 
\href{https://www.ee.iitb.ac.in/~mukul/}{\color{blue}{Prof. Mukul C. Chandorkar }}& • & Power Electronics, Power quality, Static Compensation, Motor Drives \\ 
\hline 
\href{https://www.ee.iitb.ac.in/wiki/faculty/kishore}{\color{blue}{Prof. Kishore Chatterjee}} & • & Utility friendly converter topologies, Power Factor Correction techniques,STATCOM, Switched Mode Rectifiers, Electronic Ballast \\ 
\hline 
\href{https://www.ee.iitb.ac.in/wiki/faculty/bgf}{\color{blue}{Prof. Baylon G. Fernandes}} & • & Inverter topologies for VAR compensation, Power electronic interface for non-conventional energy sources, Permanent magnet machines for
wind power generation, Switched reluctance machines for electric vehicle application \\ 
\hline 
\href{https://www.ee.iitb.ac.in/wiki/faculty/sak}{\color{blue}{Prof. Shrikrishna A. Khaparde }}& • & Deregulation in Power Industry: optimal bidding, and congestion management,Object Oriented Power System Analysis, Controlled series compensation using SSSC, Harmonic Distortion in Distribution systems, Design and Operation of small tidal power plant, Modeling and Design of transformer \\ 
\hline 
\href{https://www.ee.iitb.ac.in/wiki/faculty/svk}{\color{blue}{Prof. Shrikrishna V. Kulkarni }}& • & Transformers: Design, Analysis and Diagnostics, Electromagnetic and Coupled Field Computations, Power Engineering: Distributed Generation, High Voltage Engineering: Insulation Design/Diagnostics \\ 
\hline 
\href{https://www.ee.iitb.ac.in/wiki/faculty/anil}{\color{blue}{Prof. Anil Kulkarni }}& • & Power System Dynamics and Control, Application of Power Electronics to Power Systems, Flexible AC Transmission Systems \\ 
\hline 
\href{https://www.ee.iitb.ac.in/~pcpandey/}{\color{blue}{Prof. Prem C. Pandey}} & • & Speech and Signal Processing, Biomedical Signal Processing and Instrumentation,Electronic Instrumentation, Embedded Electronic System Design \\ 
\hline 
\href{https://www.ee.iitb.ac.in/wiki/faculty/ashukla}{\color{blue}{Prof. Anshuman Shukla }}& • & Multilevel converters and Modulation and control of power electronic converters, Power electronics applications in power systems (FACTS,HVDC, custom power devices, etc.), Renewable energies and Energy storage, Control of electric drives, Hybrid and solid-state circuit breakers and current limiters \\ \hline 
\href{https://www.ee.iitb.ac.in/wiki/faculty/soman}{\color{blue}{Prof. Shreevardhan A. Soman}} & • & Power system analysis, computation and economics, Power system protection \\ 
\hline 
\href{https://www.ee.iitb.ac.in/web/faculty/homepage/anu}{\color{blue}{Prof. Anupama Kowli }} & • & Power System Planning, Operations and Control, Electricity Markets and Economics of Electric Power Grids, Demand-side Management, Demand
Response and Flexible Loads, Smart Grids and its Enabling Technologies and Mechanisms, Policy and Regulation for Electric Power Grids. \\ 
\hline 
\href{https://www.ee.iitb.ac.in/web/faculty/homepage/hjbahirat}{\color{blue}{Prof. Himanshu J. Bahirat }}& • & Renewable Energy Sources; Grid Integration of Renewable Energy; Offshore Wind Energy; Transients in Power Systems; DC Power Systems;
DC Wind Farms; Multi-terminal DC Networks; Circuit Breakers; Power Electronics. \\ 
\hline 
\end{tabular} 
\section{EE5: Electronic Systems}
\begin{tabular}{p{3.5cm} p{3.5cm}p{9cm}}
\hline 
\hline 
Faculty  & Office  & Research Interests \\ 
\hline
\href{https://www.ee.iitb.ac.in/~madhav/}{\color{blue}{Prof. Madhav P. Desai}} & • & VLSI Circuits and Systems, VLSI design and design automation,Graph theory and combinatorics.\\ 
\hline 
\href{https://www.ee.iitb.ac.in/web/people/faculty/home/sdgupta}{\color{blue}{Prof. Siddhartha P. Duttagupta }}& • & Microelectronics, Micro/Nano Sensor Technology Optimization and Application, Sensor Integrated Electronic Circuits and Systems- Design \\ 
\hline 
\href{https://www.ee.iitb.ac.in/wiki/faculty/vmgadre}{\color{blue}{Prof. Vikram M. Gadre}} & • & Communication and signal processing, with emphasis on multiresolution and multi-rate signal processing, especially wavelets and filter banks: theory and applications \\ 
\hline 
\href{https://www.ee.iitb.ac.in/wiki/faculty/shalabh}{\color{blue}{Prof. Shalabh Gupta }}& • & High-speed CMOS analog/RF/mm-wave integrated circuits and systems,Optical fiber communication systems, Microwave photonics / ultrafast data  conversion using photonics , Beam forming antenna systems, Signal processing for these systems \\ 
\hline 
\href{https://www.ee.iitb.ac.in/wiki/faculty/jjohn}{\color{blue}{Prof. Joseph John}} & • & Analog and Digital Circuits, Optical Fiber Communications, Indoor Optical Wireless Systems, Modern Electronic Systems and Instrumentation \\ 
\hline 
\href{https://www.ee.iitb.ac.in/wiki/faculty/merchant}{\color{blue}{Prof. Shabbir Merchant }}& • & Signal Processing, Adaptive Signal Processing\\ 
\hline 
\href{https://www.ee.iitb.ac.in/~pcpandey/}{\color{blue}{Prof. Prem C. Pandey}} & *  &  Speech and Signal Processing, Biomedical Signal Processing and Instrumentation, Electronic Instrumentation, Embedded Electronic System Design\\ 
\hline 
\href{https://www.ee.iitb.ac.in/wiki/faculty/patkar}{\color{blue}{Prof. Sachin Patkar}} & • & Combinatorial optimization Matroid Theory Submodular Functions
Linear/Integer programming Network Flows High Performance Computing FPGA-based accelerated computing GPU based acceleration High Performance Circuit Simulation Algorithms Design and Analysis \\ 
 \hline 
 \href{https://www.ee.iitb.ac.in/wiki/faculty/prao}{\color{blue}{Prof. Preeti Rao}} & • &  Speech and Audio Signal Processing, Music Information Retrieval \\ 
 \hline 
 \href{https://www.ee.iitb.ac.in/wiki/faculty/dinesh}{\color{blue}{Prof. Dinesh K. Sharma}} & • & MOS device modeling VLSI design and technology. Microelectronics - technology and device characterization mixed signal design \\ 
 \hline 
 \href{https://www.ee.iitb.ac.in/~viren/}{\color{blue}{Prof. Virendra Singh}} & • & Computer Architecture Processor architecture and micro-architecture
VLSI Testing Fault-tolerant computing Robust design and architectures Self-healing system design SoC/NoC design and test Post Silicon Debug High level synthesis Formal verification. \\ 
 \hline 
  \href{https://www.ee.iitb.ac.in/wiki/faculty/mshojaei}{\color{blue}{Prof. Maryam Shojaei Baghini}} & • & Analog/Mixed-signal VLSI design and test (SoC, LV, LP, LE, Bio-
medical/Biosensors, Bio-inspired circuits and systems, I/O, highly precise circuits and systems, instrumentation, energy harvesting and many more applications), Specific technologies and performance-optimized Analog/mixed-signal/RF circuits and systems for healthcare applications.\\ 
\hline 
 \href{https://www.ee.iitb.ac.in/web/people/faculty/home/rajeshzele}{\color{blue}{Prof. Rajesh H. Zele }}& • & RF, Analog and Mixed-Signal Circuits for Communication Applications. \\ 
\hline 
 \href{http://www.ee.iitb.ac.in/~stallur/index.php}{\color{blue}{Prof. Prof. Siddharth Tallur}} & • &  RF MEMS, Photonics, Opto-Mechanics, Micro- and Nano-fabrication, Sensor Systems.\\ 
\hline 
\end{tabular} 
\section{EE6: Integrated Circuits}
\section{EE7: Solid State Devices} 
\chapterimage{./pictures/dep_act_bg.jpg}
\chapter{Department Activities}
\section {Student's Reading Group (SRG):}
The SS Students’ Reading Group (SRG) was started in 2015 as an interactive peer review basedplatf orm to share knowledge and research issues from various domains of Electrical Engineering. The 5 specializations of EE department are divided into 4 clusters and researchers from every cluster present and discuss their work with their peer in the talks that are conducted throughout the semester. The sessions are entirely student run, giving the speakers a unique opportunity to present their ideas freely and receive reviews from the students alone.\newline
New phase of SRG begins every semester. A 5 minute research challenge (5MR) is conducted in each phase where the participants get 5 minute to put forward their research ideas. A panel of faculty members decides the best speakers who are then awarded. The 5MR challenge is aimed
at improving the technical communication skills of the student.
\section {Department Academic Assistance Program (DAAP):}
Department Academic Assistance Program (DAAP) is a helping hand to the students who are facing challenges in their academics. A student can face academic related issues due to various reasons, like unacquaintance with highly competitive environment at IIT, managing the academic
workload and assignment timelines, unclear about field of interest etc. We believe that a little help from an experienced friend can make a difference. We provide one to one assistance through regular meetings. Assistants are student who have already excelled in the concerned field/course. They can complement classroom learning by providing techniques to manage course content effectively and sharing the resources. Since the inception of this program we have observed an improvement in the performance of more than 15 students. DAAP is creating a better environment for knowledge sharing and learning with joy!
\newline
\section {IEEE Students Chapter:}
The IIT Bombay IEEE Students’ Chapter is a student body that strives to promote excellence in various fields of electrical engineering. The organization is involved in organizing talks, workshops and other activities to help students obtain new skills in their fields of interest.\newline
\section {Bridge Course:}
The Bridge Course, starting from 2016 is an initiative by the department to help new students of the M. Tech. and PhD programs which makes them comfortable with their coursework. The Bridge course focuses on revising essential prerequisites and developing analytical skill. It is felt that these two essential skills will help students to work on their courses more effectively.\newline
\section {AAVRITI (Department Techfest):}
It is the annual research and technological festival of Electrical Engineering department of IIT Bombay. AAVRITI, with the motto of promoting technology, creativity, intelligence and sheer innovation, and to enthrall the magical impact of electronics that it has on the human civilization over the years aims to bridge the gap between industry \& academia by encouraging exchange of ideas and providing opportunities for technical interactions.\newline
It includes workshops on most advanced and buzzing technologies, lecture series by experts of particular fields as well as competitions and hackathons. This gives students a chance to expand their horizons and learn beyond engineering syllabus.\newline
Other department activities include Teacher’s day, Sports day, Department trip, Convocation ceremony, Valedictory function and so on.
\chapterimage{./pictures/dept_loc.jpg}
\chapter{Department Location }
Our department is large in terms of real estate. Our department is spread over various buildings: \newline
\begin{enumerate}
    \item  \textbf{EE Building:} \newline  Behind the SJMSOM building. It Resides PC lab, WEL, PEPS labs, Signal processing and instrumentation lab.
    \item  \textbf{Girish Gaitonde Building:} \newline Referred as GG Building mainly. Situated behind SJMSOM and connected to EE building. Resides EE office, Class rooms, VLSI design lab, Department Library, Embedded Systems Lab.
    \item \textbf{Annex Building:}\newline  Situated across infinity corridor behind EE building. It resides Fabrication lab facilities.
    \item \textbf{CEN Building:} \newline Connected to annex building. Resides characterization labs. Faculty offices.
\end{enumerate}

%----------------------------------------------------------------------------------------
%	CHAPTER 2
%----------------------------------------------------------------------------------------
\chapterimage{./pictures/iscp_dep_bg.png}

\chapter{Discovering what to do...}

IIT Bombay provides a multitude of options for discovering yourself. Engage and explore . All the best
\begin{remark}
	explore more at \href{http://www.iitb.ac.in/en/activities/student-clubs}{\color{blue}{Clubs at IITB}}
\end{remark}
\newpage
~\vfill
\thispagestyle{empty}

%----------------------------------------------------------------------------------------
%	CHAPTER 4
%----------------------------------------------------------------------------------------

\chapterimage{./pictures/logos.jpg} % Chapter heading image
\chapter{Department Student Representatives}

\section{Department placement coordinators}\index{Department placement coordinators}\par

\begin{center}
	\begin{tabular}{ccc}
		  \photo{1cm}{35mm}{./dep_sr/saurav.jpg}
		& \photo{1cm}{35mm}{./dep_sr/sourabh_suri.jpg}
		& \photo{1cm}{35mm}{./dep_sr/jay.jpg}\\
		Sourabh & Saurabh Suri & Jay Adhadhuk\\
		rand@gmail.com & change\_this@gmail.com & jay@gmail.com \\
	\end{tabular}
\end{center}


\sectionlinetwo{magenta}{85}

\section{Company coordinators}\index{Company coordinators}

\begin{center}
	\begin{tabular}{cccc}
		  \photo{1cm}{35mm}{./dep_sr/saurav.jpg}
		& \photo{1cm}{35mm}{./dep_sr/sourabh_suri.jpg}
		& \photo{1cm}{35mm}{./dep_sr/rahul.jpg}
		& \photo{1cm}{35mm}{./dep_sr/jay.jpg}\\
		Sourabh & Saurabh Suri & Rahul CP & Jay Adhadhuk\\
		rand@gmail.com & rand2@gmail.com & rand3.gmail.com
		& jay@gmail.com \\
	\end{tabular}
\end{center}

\sectionlinetwo{magenta}{85}
 \newpage
\section{EE Student Assosciation}\index{EE Student Assoscition}


Posts relevant to a fresher Mtech student are mentioned here. For more information about EESA visit \href{https://www.ee.iitb.ac.in/course/~eesa}{\color{blue}{EESA website}}


\begin{center}
	\begin{tabular}{ccc}
		  \photo{1cm}{35mm}{./dep_sr/saurav.jpg}
		& \photo{1cm}{35mm}{./dep_sr/sourabh_suri.jpg}
		& \photo{1cm}{35mm}{./dep_sr/jay.jpg}\\
		  \textbf{SRG Overall-Cordinator} 
		& \textbf{General Secretary}
		& \textbf{Cultural Secretary}\\
		Sourabh & Saurabh Suri & Jay Adhadhuk\\
		rand@gmail.com & dgsec@ee.iitb.ac.in & jay@gmail.com \\
	\end{tabular}
\end{center}

\begin{center}
	\begin{tabular}{cc}
		  \photo{1cm}{35mm}{./dep_sr/prashant2.jpg}
		& \photo{1cm}{25mm}{./dep_sr/prashant2.jpg}\\
		  \textbf{Sports Secretary (Boys)} 
		& \textbf{Sports Secretary (Girls)}\\
		Prashant Sharma & BVS Anusha \\
		183079037@iitb.ac.in & bvsanusha@ee.iitb.ac.in  \\
	\end{tabular}
\end{center}
\sectionlinetwo{magenta}{88}
\newpage
\section{EE ISCP Team}\index{EE ISCP Team}
\begin{center}
	\begin{tabular}{ccc}
		\photo{1cm}{35mm}{./iscp/sitaram.jpg}
		& \photo{1cm}{32mm}{./iscp/indrani.jpg}
		& \photo{1cm}{35mm}{./iscp/tarun.jpg}\\
		  \textbf{G Nagasitaram}
		& \textbf{Indrani Mukherjee}
		& \textbf{Tarun S} \\
		\textit{sitaram@ee.iitb.ac.in}
		&\textit{mukherjee.indrani22@ee.iitb.ac.in}
		&\textit{tarunsathesh@gmail.com}\\
		\photo{1cm}{31mm}{./iscp/aswin.jpg}
		& \photo{1cm}{35mm}{./iscp/sourabh.jpg}
		& \photo{1cm}{35mm}{./iscp/pankaj.jpeg}\\
		   \textbf{Aswin Ajayan}
		&  \textbf{Patil  Sourabh}
		&  \textbf{Pankaj Singh}\\
		\textit{aswin@ee.iitb.ac.in}
		&\textit{sourabh.p.iitb@gmail.com}
		&\textit{contact.pankaj.singh7@gmail.com}\\
		\photo{1cm}{35mm}{./iscp/amey.jpg}
		& \photo{1cm}{30mm}{./iscp/raman.jpg}
		& \photo{1cm}{35mm}{./iscp/yesh.jpeg}\\
		  \textbf{Chindarkar Amey}
		& \textbf{Raman Thukral} 
		& \textbf{Yaswanth Chebrolu}\\
		\textit{amey2994@gmail.com}
		&\textit{ramanthukral111@gmail.com}
		&\textit{yaswanthe.chebrolu@gmail.com}\\
		\photo{1cm}{26mm}{./iscp/risabh.jpg}
		& \photo{1cm}{35mm}{./iscp/jahnavi.jpg}
		& \photo{1cm}{35mm}{./iscp/sabitha.jpg}\\
		  \textbf{Risabh Chana}
		& \textbf{N Jahnavi}
		& \textbf{Sabitha Joseph}	\\	 
		\textit{rishabhchana844@gmail.com}
		&\textit{jahnavireddy924@gmail.com}
		&\textit{sabi.joseph3@gmail.com}\\
		\photo{1cm}{35mm}{./iscp/ashvini.jpg}
		& \photo{1cm}{27mm}{./iscp/pulkit.jpg}
		& \photo{1cm}{30mm}{./iscp/nijil.jpg}\\
		  \textbf{Ashvini Kumar}
		& \textbf{Pulkit Jain}
		& \textbf{Nijil George}	\\	 
		\textit{sharmaashvinikumar8@gmail.com}
		&\textit{jainpulkit54@gmail.com}
		&\textit{nijiliitb@gmail.com}\\

	\end{tabular}
\end{center}

\sectionlinetwo{magenta}{88}

\subsection{Links you should check out}
\href{https://insti.app/feed}{\color{blue}{Insti App}} Know your Campus
\textit{Wish you all the best, ISCP Team}
\end{document}
